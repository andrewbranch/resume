%% Copyright 2006-2013 Xavier Danaux (xdanaux@gmail.com).
%
% This work may be distributed and/or modified under the
% conditions of the LaTeX Project Public License version 1.3c,
% available at http://www.latex-project.org/lppl/.


\documentclass[10pt,a4paper,sans]{moderncv}        % possible options include font size ('10pt', '11pt' and '12pt'), paper size ('a4paper', 'letterpaper', 'a5paper', 'legalpaper', 'executivepaper' and 'landscape') and font family ('sans' and 'roman')
\usepackage{helvet}
\newcommand{\light}[1]{\textcolor{gray}{#1}}


% moderncv themes
\moderncvstyle{classic}                             % style options are 'casual' (default), 'classic', 'oldstyle' and 'banking'
\moderncvcolor{grey}                               % color options 'blue' (default), 'orange', 'green', 'red', 'purple', 'grey' and 'black'
%\renewcommand{\familydefault}{\sfdefault}         % to set the default font; use '\sfdefault' for the default sans serif font, '\rmdefault' for the default roman one, or any tex font name
%\nopagenumbers{}                                  % uncomment to suppress automatic page numbering for CVs longer than one page

% character encoding
%\usepackage[utf8]{inputenc}                       % if you are not using xelatex ou lualatex, replace by the encoding you are using
%\usepackage{CJKutf8}                              % if you need to use CJK to typeset your resume in Chinese, Japanese or Korean

% adjust the page margins
\usepackage[scale=0.75, top=2cm, bottom=1cm]{geometry}
%\setlength{\hintscolumnwidth}{3cm}                % if you want to change the width of the column with the dates
%\setlength{\makecvtitlenamewidth}{10cm}           % for the 'classic' style, if you want to force the width allocated to your name and avoid line breaks. be careful though, the length is normally calculated to avoid any overlap with your personal info; use this at your own typographical risks...

% personal data
\name{Andrew}{Branch}
%\title{Resumé title}                               % optional, remove / comment the line if not wanted
%\address{101 Polk St}{Unit 910}{San Francisco, CA 94102}
%\phone[mobile]{+1~(234)~567~890}                   % optional, remove / comment the line if not wanted; the optional "type" of the phone can be "mobile" (default), "fixed" or "fax"
%\phone[fax]{+3~(456)~789~012}
\email{andrew.branch@microsoft.com}                               % optional, remove / comment the line if not wanted
%\homepage{www.johndoe.com}                         % optional, remove / comment the line if not wanted
\social[github]{andrewbranch}
\social[linkedin]{andrewcbranch}                        % optional, remove / comment the line if not wanted
\social[twitter]{atcb}                             % optional, remove / comment the line if not wanted
%\extrainfo{additional information}                 % optional, remove / comment the line if not wanted
%\photo[64pt][0.4pt]{picture}                       % optional, remove / comment the line if not wanted; '64pt' is the height the picture must be resized to, 0.4pt is the thickness of the frame around it (put it to 0pt for no frame) and 'picture' is the name of the picture file
%\quote{Some quote}                                 % optional, remove / comment the line if not wanted

\nopagenumbers

% to show numerical labels in the bibliography (default is to show no labels); only useful if you make citations in your resume
%\makeatletter
%\renewcommand*{\bibliographyitemlabel}{\@biblabel{\arabic{enumiv}}}
%\makeatother
%\renewcommand*{\bibliographyitemlabel}{[\arabic{enumiv}]}% CONSIDER REPLACING THE ABOVE BY THIS

% bibliography with mutiple entries
%\usepackage{multibib}
%\newcites{book,misc}{{Books},{Others}}
%----------------------------------------------------------------------------------
%            content
%----------------------------------------------------------------------------------
\begin{document}
%-----       resume       ---------------------------------------------------------
\makecvtitle
\vspace*{-2\baselineskip}
\raggedright
I'm a frontend engineer with a passion for making other people's lives easier. I love designing APIs that are intuitive but powerful. Writing good documentation is an art I pursue. Technologies that let me write more expressive code excite me. I love collaborating and exchanging knowledge with others, and I'm constantly looking for a challenge.

\section{Work Experience}
	\cventry{July~2016--Present}{Software Engineer}{Microsoft}{San Francisco, CA}{}{
	\begin{itemize}%
		\item Built and designed APIs for a complete React component library forming the basis of Visual Studio App Center's UI
		\item Pair programmed with feature teams to help them successfully integrate shared components into product code
		\item Wrote API documentation and usage guides, created example apps, produced instructional videos, and gave online trainings to outline best practices for frontend architecture with React, MobX, and CSS Modules
		\item Contributed to and authored several open-source packages focused on increasing developer productivity
		\item Worked side-by-side with designers and product managers to deliver user-facing features
		\item Owned App Center's frontend development experience, from keeping Webpack running fast to integrating an automated browser testing solution
	\end{itemize}}

	\cventry{March~2015--June~2016}{Full Stack Web Developer}{Xamarin}{San Francisco, CA}{}{
	\begin{itemize}%
		\item Created web apps enabling self-service updates to xamarin.com by various teams
		\item Created and managed web infrastructure to improve scalability, reliability, performance, and developer workflow for sites with millions of visitors
		\item Worked with the marketing team to track, optimize, and report on user acquisition and retention
		\item Helped lead Xamarin to an acquisition by Microsoft in March 2016
	\end{itemize}}

	\cventry{October~2014--January~2015}{iOS + Web Developer}{Freelance work}{Pensacola, FL}{}{
	\begin{itemize}%
		\item Designed and built a production-ready iPhone app for multiple screen sizes with Swift
		\item Designed an attractive icon set for use within the app
		\item Used Node.js and Ember.js to build an admin web app for managing the iOS app's data
	\end{itemize}}

	\cventry{August~2013--May~2014}{Full Stack Web Developer}{Auburn University}{Auburn, AL}{}{
		\begin{itemize}%
			\item Interfaced with clients to generate list of technical requirements and project specifications
			\item Wrote data-heavy web applications using C\# and ASP.NET MVC
			\item Leveraged HTML, JavaScript, and CSS to create intuitive and responsive user interfaces
		\end{itemize}}

\section{Key Skills}
	\begin{minipage}[t]{0.5\textwidth}
		\cvitem{Frameworks~\& languages}{\small
			\begin{itemize}
				\item React
				\item TypeScript
				\item Redux
				\item MobX
				\item CSS \light{(modules, in-JS, Sass)}
			\end{itemize}
		}
		\cvitem{Techniques~\& concepts}{\small
			\begin{itemize}
				\item Web accessibility
				\item Unit \& snapshot testing \light{(Jest)}
				\item Browser testing \light{(TestCafe, Puppeteer)}
				\item Performance tuning
			\end{itemize}
		}
	\end{minipage}
	\begin{minipage}[t]{0.3\textwidth}
		\vspace{-38pt}
		\cvitem{Dev ops}{\small
			\begin{itemize}
				\item Webpack \light{(speed, bundle splitting)}
				\item VSTS pipelines
				\item AWS, Azure deployments
				\item Docker
			\end{itemize}
		}
	\end{minipage}

\section{Education}
	\cventry{2010--2014}{Bachelor of Electrical Engineering}{Auburn University}{Auburn, AL}{}{}  % arguments 3 to 6 can be left empty

% Publications from a BibTeX file without multibib
%  for numerical labels: \renewcommand{\bibliographyitemlabel}{\@biblabel{\arabic{enumiv}}}% CONSIDER MERGING WITH PREAMBLE PART
%  to redefine the heading string ("Publications"): \renewcommand{\refname}{Articles}
\nocite{*}
\bibliographystyle{plain}
\bibliography{publications}                        % 'publications' is the name of a BibTeX file

% Publications from a BibTeX file using the multibib package
%\section{Publications}
%\nocitebook{book1,book2}
%\bibliographystylebook{plain}
%\bibliographybook{publications}                   % 'publications' is the name of a BibTeX file
%\nocitemisc{misc1,misc2,misc3}
%\bibliographystylemisc{plain}
%\bibliographymisc{publications}                   % 'publications' is the name of a BibTeX file

\end{document}
